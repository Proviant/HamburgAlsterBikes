% The section about the used in- and output configuration for the simulation
% @author Kalvin Döge
%


\section{Konfiguration der Ein- und Ausgabedaten}\label{sec:input-output-configuration}

Nun folgen die Eingabedaten, die bei der Implementation in die Simulation benötigt werden:

\begin{itemize}
    \item \textbf{SpatialMediatorGraph-Konfiguration:} Für die Simulationsumgebung ist eine \code{GeoJSON} vorgesehen, der den Simulations- und Bewegungsbereich der Agenten und Entitäten angibt.
    \item \textbf{HumanTraveler-Konfiguration:} Die aktiven \code{HumanTraveler} und \code{BicycleLeader} in der Simulation benötigen folgende Eingabedaten als \code{CSV}-Datei:
    \begin{itemize}
        \item \code{startTime} als Startuhrzeit in dem Format ,,HH:mm``, sobald die Erstellungsphase für \code{HumanTraveler} beginnt
        \item \code{endTime} als Enduhrzeit in dem Format ,,HH:mm``, sobald die Erstellungsphase der \code{HumanTraveler} aufhört
        \item \code{spawningIntervalInMinutes} als Ganzzahlangabe für die Wiederholungen der Erstellungen pro angegebener Minuten
        \item \code{spawningAmount} als Anzahl der zu erstellenden Agenten
        \item \code{Wahrscheinlichkeiten im Bereich \code{0 <= hasBike <= 1} für:}
        \begin{itemize}
            \item \code{hasBike} zum Besitz eines eigenen Fahrrads
            \item \code{hasCar} zum Besitz eines eigenen Pkws, das höher als \code{hasBike} sein wird um den Verkehr Pkw-dominant zu machen und näher an der Realität zu simulieren
            \item \code{prefersBike} zum Bevorzugen eines Fahrrads
            \item \code{prefersCar} zum Bevorzugen eines Pkws, das höher als \code{prefersBike} sein wird um den Verkehr Pkw-dominant zu machen und näher an der Realität zu simulieren
            \item \code{usesBikeAndRide} zum Nutzen von \code{RentalBicycle}s
            \item \code{usesOwnBikeOutside} zum Nutzen von dem eigenen Fahrrad
            \item \code{usesOwnCar} zum Nutzen eines eigenen Pkws
        \end{itemize}
        \item \code{source} als Startbereich, in dem der \code{HumanTraveler} erstellt wird
        \item \code{destination} als Zielbereich, in dem der \code{HumanTraveler} ein Ziel auswählen kann
        \item \code{discriminator} als Unterscheidungszahl für interne Darstellungen
    \end{itemize}
    \item \textbf{TrafficLight-Konfiguration:} Die zu erschaffenden \code{TrafficLight}s benötigen zur Erstellung folgende Eingabedaten als \code{CSV}-Datei:
    \begin{itemize}
        \item \code{Longitude} als Longitude auf einer Weltkarte oder Simulationsumgebung
        \item \code{Latitude} als Latitude auf einer Weltkarte oder Simulationsumgebung
        \item \code{LengthPhaseRed} als Phasenlänge für die Lichtphase Rot
        \item \code{LengthPhaseGreen} als Phasenlänge für die Lichtphase Grün
    \end{itemize}
    \item \textbf{Simulations-Konfiguration:} Eine \code{JSON}, die alle für die Simulation relevanten Layer, Entitäten und Agenten angegeben braucht.
\end{itemize}

Zudem werden noch folgende Ausgabedaten bei der Simulation erstellt:

\begin{itemize}
    \item \textbf{\code{HumanTraveler}s Strecken-GeoJSON:} Die gefahrenen Strecken der \code{HumanTraveler} mit \code{Longitude-Latitude}-Positionsangaben
    \item \textbf{\code{Bicycle-Leader}s Strecken-GeoJSON:} Die gefahrenen Strecken der \code{BicycleLeader} mit \code{Longitude-Latitude}-Positionsangaben
    \item \textbf{Erfolgsausgabe des \code{Bicycle-Leader}s:} Wenn eine erfolgreiche ,,grüne Welle`` stattfand oder ob er durch eine Lichtsignalanlage aufgehalten wurde mit der jeweiligen Stunde der Simulation
\end{itemize}
