% The section about the projects limits
% @author Kalvin Döge
%


\section{Umfang der Arbeit}\label{sec:delimitation}

Dieser Abschnitt beschäftigt sich mit den Grenzen dieser Arbeit und Simulation.

Dadurch, dass im realen Leben Lichtsignalschaltungen verschiedenster Sicherheitsvorkehrungen unterliegen müssen, damit im Verkehr Sicherheit für alle Verkehrsteilnehmer gewährleistet werden kann, wäre eine Einbindung dieser Vorkehrungen für die Simulation ein wichtiger Realitätsaspekt.
Doch aufgrund von fehlenden Daten und genauer Auskunft der Stadt Hamburg, welche Entscheidungen Lichtsignalanlagen treffen beziehungsweise nach welchem Design sie entwickelt wurden, um die Sicherheit zu ermöglichen, fällt auch dieser Realitätsaspekt aus der Simulation aus.

Zudem werden viele Umwelteinflüsse nicht erst in der Simulation eingebaut: Beispielsweise werden Ereignisse wie Wetter, Unfälle, Baustellen oder verschiedene Verkehrsteilnehmertypen, etwa wie zu langsam Fahrende oder stetig die Spur wechselnde Teilnehmer, nicht berücksichtigt.

Die für diese Simulation genutzten Werte, wie zum Beispiel die der gesamten Agentenzahl oder die der Lichtsignalphasenlänge, sind ebenfalls aufgrund fehlender Statistiken oder Daten nicht sehr realitätsnah.
Beispielsweise wäre es angemessener, jede Lichtsignalschaltung auf der für \code{BicycleLeader} vorgesehenen Route mehrere hunderte Male zu beobachten, nur um dann eine aussagekräftigere Zeit von Rot- und Grünphasen tätigen zu können, damit statistische Abweichungen nicht potenziell das Endergebnis beeinflussen könnten.

Die Realität müsste für die Simulation über die Daten, Umwelteinflüssen und Agententypen abgebildet werden, damit ein praktischer Nutzen für die Stadt Hamburg entstehen könnte.
Entsprechend bleibt der Nutzen dieser Arbeit für die Stadt Hamburg im theoretischen Bereich.
