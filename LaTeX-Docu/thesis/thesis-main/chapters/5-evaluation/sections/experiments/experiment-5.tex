% The subsection about the results of the fifth experiment
% @author Kalvin Döge
%

\subsection{Experiment 5: 350 g | 4 r}\label{subsec:experiment-5}

In dem fünften Experiment wird nun gegen die Zeitschaltung des vierten Experiments geprüft:
Die Phasenlängen von \code{467 g | 3 r} werden durch den korrekteren, vorher berechneten Wert für die Berechnung ausgetauscht:  \code{560 g | 2,5 r}.
Dies dient dazu, die Methodik nicht durch Ungenauigkeit in der Rundung fälschlich zu gebrauchen.
Mit den eigentlichen Phasen von \code{560 g | 2,5 r} wird zuerst die Rotphase mit \code{3/2} multipliziert.
Die erhaltene Rotphasenlänge beträgt folglich \code{3,75}, die wegen der nur in ganzen Sekunden simulierbaren Zeit auf 4 Sekunden aufgerundet werden muss.
Mit dem Dreisatz~\ref{fig:dreisatz} wird dann die Grünphasenlänge berechnet, \code{350 g | 4 r}, und die Zeiten werden dann in die Experimente eingegeben:

\begin{table}[htb]
    \centering
    \begin{tabular}{||c|c|c|c|c|c|c|c|c|c|c|c||}
        \hline
        Stunde  & \#NHT  & E1  & E2  & E3  & E4  & E5  & E6  & E7  & E8  & E9  & E10 \\\hline\hline
        0       & 14     & \qg & \qg & \qg & \qg & \qg & \qg & \qg & \qg & \qg & \qg \\\hline
        1       & 51     & \qg & \qg & \qg & \qg & \qg & \qg & \qg & \qg & \qg & \qg \\\hline
        2       & 130    & \qg & \qg & \qg & \qg & \qg & \qg & \qg & \qg & \qg & \qg \\\hline
        3       & 274    & \qg & \qg & \qr & \qg & \qg & \qg & \qg & \qg & \qg & \qg \\\hline
        4       & 487    & \qg & \qg & \qg & \qr & \qg & \qg & \qg & \qg & \qg & \qg \\\hline
        5       & 707    & \qg & \qr & \qg & \qg & \qg & \qg & \qr & \qg & \qg & \qg \\\hline
        6       & 857    & \qg & \qg & \qg & \qg & \qg & \qg & \qg & \qg & \qg & \qg \\\hline
        7       & 879    & \qg & \qg & \qg & \qr & \qg & \qg & \qg & \qg & \qg & \qg \\\hline
        8       & 804    & \qg & \qg & \qg & \qg & \qg & \qg & \qg & \qg & \qr & \qg \\\hline
        9       & 684    & \qg & \qg & \qg & \qr & \qg & \qg & \qg & \qg & \qg & \qg \\\hline
        10      & 570    & \qg & \qr & \qg & \qg & \qg & \qg & \qg & \qg & \qg & \qg \\\hline
        11      & 514    & \qg & \qg & \qg & \qg & \qg & \qg & \qg & \qg & \qg & \qg \\\hline
        12      & 550    & \qg & \qg & \qg & \qg & \qg & \qr & \qg & \qg & \qg & \qg \\\hline
        13      & 650    & \qg & \qg & \qg & \qg & \qg & \qg & \qg & \qg & \qr & \qg \\\hline
        14      & 765    & \qg & \qg & \qg & \qr & \qg & \qg & \qg & \qg & \qg & \qg \\\hline
        15      & 850    & \qg & \qg & \qg & \qg & \qg & \qg & \qg & \qg & \qg & \qg \\\hline
        16      & 857    & \qg & \qg & \qg & \qg & \qg & \qg & \qg & \qg & \qg & \qg \\\hline
        17      & 755    & \qg & \qg & \qg & \qg & \qg & \qg & \qg & \qg & \qg & \qg \\\hline
        18      & 579    & \qg & \qg & \qg & \qg & \qr & \qg & \qg & \qg & \qg & \qg \\\hline
        19      & 379    & \qg & \qg & \qg & \qg & \qg & \qg & \qg & \qg & \qg & \qg \\\hline
        20      & 205    & \qg & \qg & \qg & \qg & \qg & \qg & \qg & \qg & \qg & \qg \\\hline
        21      & 94     & \qg & \qg & \qg & \qg & \qg & \qg & \qg & \qg & \qg & \qg \\\hline
        22      & 34     & \qg & \qg & \qg & \qg & \qg & \qg & \qg & \qg & \qg & \qg \\\hline
        23      & 14     & \qg & \qg & \qg & \qg & \qg & \qg & \qg & \qg & \qg & \qg \\\hline\hline
        Gesamt: & 11.703 & \qs & \qs & \qs & \qf & \qs & \qs & \qs & \qs & \qs & \qs
    \end{tabular}
    \caption{Erfolgstabelle des 5. Experimentes}
    \label{tab:experiment-5-table}
    \centering
\end{table}

Die letzte Experimententabelle~\ref{tab:experiment-5-table} stellt dar, wie bei 240 Simulationen nur 12 keine und 228 eine ,,grüne Welle`` erreichen konnten.
Der Durchschnitt beträgt dabei ungefähr 1 fehlgeschlagenes Experiment pro simulierten Tag:

\[(0 + 2 + 1 + 4 + 1 + 1 + 1 + 0 + 2 + 0) / 10 = 1,2\]

Im Durchschnitt ist zwar die 90\% Erfolgsrate eingehalten, jedoch ist aus der Tabelle~\ref{tab:experiment-5-table} das Experiment vier fehlgeschlagen.
Diese Schaltung erfüllt deshalb nur knapp nicht die geforderte Erfolgsrate an ,,grünen Wellen``.
