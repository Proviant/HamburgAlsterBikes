% !TEX root = ../thesis.tex
%
% configurations
%

% English Language support
% -> uncomment if needed
% Beta!
%\fullenglish{yes}
\fullenglish{no}

% text field
%-> replace supervisor names with correct ones
\firstSupervisor{Prof. Dr. Thomas Clemen}
\secondSupervisor{Prof. Dr. Thomas Lehmann}

% text field
%-> replace title with your thesis title
\thesisTitle{Bestimmung einer 'Grünen Welle' bei Lichtsignalschaltungen für Alster-Fahrradfahrer durch agentenbasierte Simulation mithilfe des MARS-Frameworks}
\thesisTitleEN{Determining a 'Green Wave' for traffic lights for Alster-cyclists with agent based simulation using the MARS-Framework}

% text field
%-> replace the key words with your own key words
\keywordsDE{Agentenbasierte Simulation, Alster, Hamburg, MARS-Framework, Grüne Welle, Ampelschaltung, Fahrradfahrer}
\keywordsEN{Agent based Simulation, Alster, Hamburg, MARS-Framework, Green Wave, Traffic Light Control, Cyclicsts}

% text field
%-> replace the text with a description of the thesis
\abstractDE{Fahrradfahrer haben es in Großstädten noch immer schwer, sich effizient im Straßenverkehr von Großstädten fortzubewegen. ,,Grüne Wellen`` sind für Pkw-Fahrer dabei häufiger in dem Verkehrsplan mit einbezogen, als für Fahrradfahrer. Eine Art der ,,grünen Welle`` ist dabei die Ermittlung einer Grün- und Rotphasenlänge, die über alle Ampeln hinweg gelten soll, ohne dabei eine Synchronisation untereinander zu benötigen. Dies gibt dem Verkehrsteilnehmer die subjektive Wahrnehmung, dass nur Grünphasen für ihn an Ampeln vorherrschen, ohne dabei getaktet oder von einer künstlichen Intelligenz gesteuert zu sein. Um so eine passende Zeitschaltung für Fahrradfahrer zu finden, wurde eine Rundfahrt um die Binnen- und Außenalster genommen, um dann experimentell über agentenbasierte Simulationen sich Zeitschaltung zu nähern. Im Verlauf der Experimente wird zudem gezeigt, dass diese Art ,,grünen Welle`` für die Realität zu Problemen führen wird. \dots}
\abstractEN{Cyclists still have a hard time getting around efficiently in the traffic of big cities. Green waves are more often than not included in the traffic plan decisions for car drivers, rather than for cyclists. One type of ,,green wave`` is the determination of a green and red phase length that applies across all traffic lights without requiring synchronisation between them. This gives road users the subjective perception that only green phases prevail for them at traffic lights, without being timed or controlled by an artificial intelligence. In order to find a suitable time circuit for cyclists, a round trip around the Binnen- and Außenalster was taken and then approached experimentally via agent-based simulations. In the course of the experiments, it is also shown that this kind of "green wave", if implemented in reality, will lead to problems. \dots}

% text field
%-> replace john with your name
\thesisAuthor{Kalvin Döge}

% text field
%-> enter the submission date
\submissionDate{21. September 2023}

% switch - uncomment only one
%-> uncomment NDA or public
%\NDA{yes}
\NDA{no}

% switch - uncomment only one
%-> uncomment old standard cover or cover Corporate Design 2017
\Cover{CD2017}
%\Cover{CD2017NoLogo}
%\Cover{Std2018}
%\Cover{Std2018_green} 			% with green bar

% switch - uncomment only one
%-> uncomment to show list of figures or not
\ListOfFigures{yes}
%\ListOfFigures{no}

% switch - uncomment only one
%-> uncomment to show list of tables or not
\ListOfTables{yes}
%\ListOfTables{no}

% switch - uncomment only one
%-> uncomment to show list of accronyms or not
\ListOfAccronyms{yes}
%\ListOfAccronyms{no}

% switch - uncomment only one
%-> uncomment to show list of symbols or not
\ListOfSymbols{yes}
%\ListOfSymbols{no}

% switch - uncomment only one
%-> uncomment to show list of glossary entries or not
%\Glossary{yes}
\Glossary{no}

% switch - uncomment only one
%-> uncomment the study course you are in
%\studycourse{ITS}
%\studycourse{TI}
\studycourse{AI}
%\studycourse{WI}
%\studycourse{EI}
%\studycourse{REE}
%\studycourse{BMP}
%\studycourse{BMP-hp}	 % Internship Report in M&P
%\studycourse{BMT}
%\studycourse{BMT-st}    % Study / home assignment in BMT
%\studycourse{BMT-hp}    % Internship Report in BMT
%\studycourse{MI}
%\studycourse{MIK}
%\studycourse{MA}
